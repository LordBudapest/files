\section{The Caveat Memory Model}
\label{CAVEAT}

This memory model simulates the behavior of the \textsf{Caveat}
analyser, with additional enhancements.  It is implemented as an
extension of the \texttt{Typed} memory model, with a specific
treatment of \textit{formal} variables of pointer type.

To activate
this model, simply use '\verb$-wp-model Typed+Caveat$' or
'\verb$-wp-model Caveat$' for short.

A specific detection of variables to be treated as \textit{reference
  parameters} is used.  This detection is more clever than the standard one
since it only takes into account local usage of each function (not
global ones).

%However, it can not be applied with axiomatics that
%read into the memory. Only pure logic functions and pure predicates can be
%defined in axiomatics with this model.

Additionally, the \texttt{Caveat} memory model of \textsf{Frama-C}
performs a local allocation of formal parameters with pointer
types that cannot be treated as \textit{reference parameters}.
Note that it means that the pointer is considered valid. If one needs
to accept \texttt{NULL} for this pointer, a \texttt{wp\_nullable} ghost
annotation or the clause \texttt{wp\_nullage\_args} can be used to bring
this information to WP:

\listingname{nullable.c}
\cinput{nullable.c}

This makes explicit the separation hypothesis of memory regions
pointed by formals in the \textsf{Caveat} tool. The major benefit of
the \textsf{WP} version is that aliases are taken into account by the
\texttt{Typed} memory model, hence, there are no more suspicious
\textit{aliasing} warnings.

\paragraph{Warning:} using the \texttt{Caveat} memory model,
the user \emph{must} check by manual code review that no aliases are
introduced \emph{via} pointers passed to formal parameters at call sites.

However, \textsf{WP} warns about the implicit separation hypotheses required by
the memory model \textit{via} the \texttt{-wp-warn-memory-model} option, set
by default.
