\chapter{Builtins}
\label{wp-builtins}

This chapter provides additional informations on the supported \textsf{ACSL} built-in symbols and, more generally,
on the simplification rules implemented in the \textsf{Qed} simplifier, as introduced in Chapter~\ref{wp-simplifier}.

\section{General Simplifications}

Arithmetics operators $(+,\times,\dots)$ are normalized with respect to their associativity and commutativity rules.
Neutral and absorbent operands are also taken into account, and operations on numerical constants are propagated as well.
Moreover, linear forms are strongly normalized.
Hence, it is not possible to write term $(1+x)-1+x$ since it will be automatically reduced into $2x$.

Comparison operators are also simplified. First, linear normalization is performed on both sides of the comparison,
with negative factors or constants moved on the opposite side. Finally, on integer comparison,
an off-by-one normalization is applied to transform, \emph{eg.} $1<x$ into $0\leq x$ (on integers only).

Comparison between injective functions are simplified in the expected way.
For instance, $f(x,y)=f(x',z')$ will be rewritten into $x=x' \wedge y=y'$ when $f$ is injective, and similarly with dis-equalities.

Idempotent operators $f$ are normalized with $f(x,x)=x$. Associative and commutative operators are normalized as usual
(terms are sorted structurally).

\section{ACSL Built-ins}

\newcommand{\page}[1]{\footnotesize{p.\pageref{#1}}}
\newcommand{\builtin}[1]{\mathtt{\backslash#1}}
\newcommand{\wref}[2]{\href{http://why3.lri.fr/stdlib/#1}{\texttt{#1.#2}}}

The \textsf{ACSL} built-ins supported by \textsf{Frama-C/WP} are listed in Figure~\ref{wp-builtins-list}.
In the table, the reference implementation of the function is given in terms of the \textsf{Why-3}
standard library\footnote{Available online \texttt{http://why3.lri.fr/stdlib}},
from which all the properties are automatically imported and communicated to the provers.

\begin{figure}[htbp]
\begin{center}
    \begin{tabular}{l@{\quad}lr}
    \textsf{ACSL} & \textsf{Why-3} Reference & Qed \\
    \hline
    $\mathtt{(real)}\,x$ & \wref{real}{FromInt} &\page{builtin-real} \\
    $\builtin{ceil},\builtin{floor}$ & \wref{real}{Truncate} &\page{builtin-truncate} \\
    $\builtin{abs}$ & \wref{int}{Abs}, \wref{real}{Abs} &\page{builtin-abs} \\
    $\builtin{min},\builtin{max}$ & \wref{int}{MinMax}, \wref{real}{MinMax} &\page{builtin-min-max} \\
    $\builtin{sqrt}$ & \wref{real}{Square} &\page{builtin-sqrt} \\
    $\builtin{exp}, \builtin{log}, \builtin{log10}$ & \wref{real}{ExpLog} &\page{builtin-exp-log} \\
    $\builtin{pow}$ & \wref{real}{PowerReal} &\page{builtin-pow} \\
    $\builtin{sin},\builtin{cos},\builtin{tan}$ & \wref{real}{Trigonometry} & \page{builtin-trigo} \\
    $\builtin{asin},\builtin{acos},\builtin{atan}$ & \wref{real}{Trigonometry} & \page{builtin-arctrigo} \\
    $\builtin{sinh},\builtin{cosh},\builtin{tanh}$ & \wref{real}{Hyperbolic} & \page{builtin-hyperbolic} \\
    $\builtin{atan2},\builtin{hypot}$ & \wref{real}{Polar} & \page{builtin-polar} \\
    \hline
    \end{tabular}
\end{center}
\caption{Supported \textsf{ACSL} builtins}
\label{wp-builtins-list}
\end{figure}

Below in the section, additional informations are given for each supported builtin, with a brief
description of the implemented \textsf{Qed} simplifiers.

\pagebreak
%---------------------------------------------------------------------------------------------

\vskip 1em
\hrule
\label{builtin-real}
\paragraph{$\mathtt{(real)}\,x$} for the conversion from integers to reals. Constants, including
\texttt{float} and \texttt{double} literals, are exactly converted to rationals. This operation is
also declared to be injective.

%---------------------------------------------------------------------------------------------

\vskip 1em
\hrule
\label{builtin-truncate}
\paragraph{$\builtin{ceil}(x)$, $\builtin{floor}(x)$} for the conversion from reals to integers.
Constants (rationals) are truncated towards the expected direction. Other simplifications are:
\[
\begin{array}{rcl@{\quad}l}
\builtin{ceil}(\mathtt{(real)}\,n) &=& n &(\forall n\in \mathbb{Z})\\
\builtin{floor}(\mathtt{(real)}\,n) &=& n &(\forall n\in \mathbb{Z})\\
\end{array}
\]

Other conversion simplifications will be performed when using conversions to \textsf{C}-integer types.
Actually, the truncation operation defined in \verb+real.Truncate+ from \textsf{Why-3} is used for
these other conversions and combine well together.

%---------------------------------------------------------------------------------------------

\vskip 1em
\hrule
\label{builtin-abs}
\paragraph{$\builtin{abs}(x)$} for both integers and reals. The implemented simplifiers
are:
\[
\begin{array}{rcl@{\quad}l}
\builtin{abs}(x) &\geq & 0\\
\builtin{abs}(k.x) &=& |k|.\builtin{abs}(x) \\
\builtin{abs}(x) &=& x & \mathtt{when}\; 0\leq x \\
\builtin{abs}(x) &=& -x & \mathtt{when}\; x < 0 \\
\builtin{abs}(x) &=& 0 & \mathtt{iff}\; x = 0
\end{array}
\]

%---------------------------------------------------------------------------------------------

\vskip 1em
\hrule
\label{builtin-min-max}
\paragraph{$\builtin{min}(x,y)$, $\builtin{max}(x,y)$}
are implemented on both real and integers as idempotent, associative and commutative operators.

%---------------------------------------------------------------------------------------------

\vskip 1em
\hrule
\label{builtin-sqrt}
\paragraph{$\builtin{sqrt}(x)$} is declared to be injective, monotonically increasing and is associated
with the following simplifiers:
\[
\begin{array}{rcl@{\quad}l}
\builtin{sqrt}(0) &=& 0 \\
\builtin{sqrt}(1) &=& 1 \\
\builtin{sqrt}(x) &\geq& 0 \\
\builtin{sqrt}(x) &=& 0 & \mathtt{iff}\; x = 0 \\
\builtin{sqrt}(x\times x) &=& \builtin{abs}(x) \\
\end{array}
\]

Moreover, the following lemmas are added to complement the reference ones:
\[
\begin{array}{rcl@{\quad}l}
\builtin{sqrt}(x) &<& x & \text{when}\; 1<x \\
\builtin{sqrt}(x) &>& x & \text{when}\; 0\leq x < 1 \\
\end{array}
\]

%---------------------------------------------------------------------------------------------

\vskip 1em
\hrule
\label{builtin-exp-log}
\paragraph{$\builtin{exp}(x)$, $\builtin{log}(x)$ and $\builtin{log10}(x)$} are declared
to be injective, monotonically increasing and are associated with the following simplifiers:
\[
\begin{array}{rcl@{\quad}l}
\builtin{log}(1) &=& 0\\
\builtin{exp}(0) &=& 1\\
\builtin{exp}(x) &>& 0\\
\builtin{exp}(\builtin{log}(x)) &=& x & \text{when}\; 0<x \\
\builtin{log}(\builtin{exp}(x)) &=& x \\
\end{array}
\]

However, we found necessary to complement the reference definitions with the following
lemma, especially when $\builtin{tanh}$ is involved:
\[
    \builtin{exp}(x) \geq 0
\]

%---------------------------------------------------------------------------------------------

\vskip 1em
\hrule
\label{builtin-pow}
\paragraph{$\builtin{pow}(x,n)$} is associated with the following simplifiers:
\[
\begin{array}{rcl@{\quad}l}
\builtin{pow}(x,0) &=& 1\\
\builtin{pow}(x,1) &=& x\\
\end{array}
\]

Other algebraic simplifications are limited to cases where $x>0$, and there is no obvious
choice of normalization when combining $\builtin{pow}$ with $\builtin{exp}$ and $\builtin{log}$.
\[
\begin{array}{rcl@{\quad}l}
\builtin{log}(\builtin{pow}(x,n)) &=& n.\builtin{log}(x) & \text{when}\; 0<x  \\
\end{array}
\]

However, algebraic rules for such combinations are well established in the reference lemmas
imported from \textsf{Why-3}.

%---------------------------------------------------------------------------------------------

\vskip 1em
\hrule
\label{builtin-trigo}
\paragraph{$\builtin{sin}(x)$, $\builtin{cos}(x)$ and $\builtin{tan}(x)$} trigonometric
operations. Useful lemmas are already defined in the reference implementation from
\textsf{Why-3} library.

\label{builtin-arctrigo}
\paragraph{$\builtin{asin}(x)$, $\builtin{acos}(x)$ and $\builtin{atan}(x)$} trigonometric
operations. Except definition of $\builtin{atan}$ which is available from reference,
definition for arc-$\builtin{sin}$ and arc-$\builtin{cos}$ have been added. Moreover,
all the arc-trigonometric operations are declared to be injective. In addition, the
following simplifiers are also registered:

\[
\begin{array}{rcl@{\quad}l}
\builtin{sin}(\builtin{asin}(x)) &=& x & \text{when}\; -1 \leq x \leq 1 \\
\builtin{cos}(\builtin{acos}(x)) &=& x & \text{when}\; -1 \leq x \leq 1 \\
\builtin{tan}(\builtin{atan}(x)) &=& x\\
\end{array}
\]

Notice that definitions for $\builtin{asin}$ and $\builtin{acos}$ are not (yet) available
from the original \textsf{Why-3} reference
and are actually provided by custom extensions installed in the \textsf{WP} shared directory.

%---------------------------------------------------------------------------------------------

\vskip 1em
\hrule
\label{builtin-hyperbolic}
\paragraph{$\builtin{sinh}(x)$, $\builtin{cosh}(x)$ and $\builtin{tanh}(x)$} trigonometric
operations. Useful lemmas are already defined in the reference implementation from
\textsf{Why-3} library.

%---------------------------------------------------------------------------------------------

\vskip 1em
\hrule
\label{builtin-polar}
\paragraph{$\builtin{atan2}(x,y)$ and $\builtin{hypot}(x,y)$} for dealing with
polar coordinates. Definitions imported from the reference implementation of
\textsf{Why-3} library.

%---------------------------------------------------------------------------------------------

\vskip 1em
\hrule
\label{builtin-fpcmp}
\paragraph{$\builtin{le\_float}(x,y)$, $\builtin{ge\_float}(x,y)$,
$\builtin{lt\_float}(x,y)$, $\builtin{gt\_float}(x,y)$, $\builtin{eq\_float}(x,y)$,
$\builtin{ne\_float}(x,y)$, $\builtin{le\_double}(x,y)$,
$\builtin{ge\_double}(x,y)$, $\builtin{lt\_double}(x,y)$,
$\builtin{gt\_double}(x,y)$, $\builtin{eq\_double}(x,y)$, and
$\builtin{ne\_double}(x,y)$} for dealing with floating point
comparisons. They are similar to comparisons over the real numbers if both
$x$ and $y$ are finite, but obey IEEE semantics for infinities and NaNs

%---------------------------------------------------------------------------------------------

\section{Custom Extensions}

\paragraph{DEPRECATED.} You shall import a Why3 module instead of using custom
drivers. See Section~\ref{importer} for details.

As explained in Section~\ref{drivers}, it is possible to extend all the properties mentioned
above. Section~\ref{wp-custom-tactics} also provides hints on how to define tactics that can be
used in the interactive prover of \textsf{Frama-C/WP}.

However, \textsf{Why-3} offers the easiest way of adding new theorems for interactive proving.
Since all the builtin symbols of \textsf{ACSL} are actually linked to the standard library
of \textsf{Why-3}, any user theory also referring to the standard symbols can be added to the proof
environment by using option \verb+-wp-why-lib 'file...'+.

Otherwise, a driver can be written. For instance, the additional lemma regarding $\builtin{exp}$
p.\pageref{builtin-hyperbolic} for \textsf{Why3} is simply defined in the following way:

\begin{logs}
    // MyExp.mlw file
    axiom exp_pos : forall x:real. (0.0 <  exp(x))
\end{logs}

Of course, this piece of \textsf{Why3} input file must be integrated after the symbol
\verb+exp+ has been defined. The corresponding driver is then:

\begin{logs}
    // MyExp.driver file
    library exponential:
    why3.file += "MyExp.mlw" ;
\end{logs}

Such a driver, once loaded with option \verb+-wp-driver+, instructs \textsf{Frama-C/WP} to
append your extension file to the other necessary resources for library \verb+"exponential"+, to
which the logic ACSL builtin \verb+\exp+ belongs to.

The file \verb+share/wp/wp.driver+ located into the shared directory of \textsf{Frama-C},
and the associated files in sub-directory \verb+share/wp/why3+ shall provide
all the necessary hints for extending the capabilities of \textsf{Frama-C/WP} in a similar way.
